\documentclass{article}

\usepackage[margin=4cm]{geometry}
\usepackage{fancyhdr}
    \pagestyle{fancy}
\usepackage{fontspec}
    \setsansfont{Linux Biolinum O}
\usepackage{polyglossia}
    \setmainlanguage{english}
\usepackage{sectsty}
    \sectionfont{\normalfont\sffamily\bfseries\color{blue!40!black}}
    \subsectionfont{\normalfont\sffamily\bfseries\color{blue!30!black}}
\usepackage{amsmath}
\usepackage{amssymb}
\usepackage{siunitx}
\usepackage{float}
\usepackage{booktabs}
\usepackage{subcaption}
\usepackage{graphicx}
\usepackage{xcolor}
\usepackage{listings}
    \lstset{language=Python,
	basicstyle=\footnotesize\ttfamily,
	breaklines=true,
	framextopmargin=50pt,
	frame=bottomline,
	backgroundcolor=\color{white!86!black},
	commentstyle=\color{blue},
	keywordstyle=\color{red},
	stringstyle=\color{orange!80!black}}
\usepackage{tikz}
\usepackage{hyperref}

\title{\textsf{\color{blue!40!black}Checklist for Debian System}}
\author{Maurice Donner}

\begin{document}

\maketitle

\section{Basic Linux terminal operations}
The following commands are really important and you will use them a lot
\begin{lstlisting}
ls	    # List contents of a directory
ls -lah	    # -list -all -humanreadable
man	    # Displays manual pages
cat	    # Print file in current terminal (quickly look at file contents)
pwd	    # Print the current working directory
cd	    # Change Directory, you will use this a lot
rm	    # Removes Files (forever! be careful) | rmdir for empty directories
cp	    # Copy file1 to file2 (cp file1 file2)
mkdir	    # Create a new Directory
find -name  # find a file (from the current directory)
grep	    # Find words in a file
tar -xvf    # Install Programs on linux that end on .tar.gz
chmod	    # Change file Permissions quickly
echo	    # Print a string to the terminal, or to file: echo "String" >> a.txt
\end{lstlisting}
Next, this is how you can quickly handle files in linux
\begin{lstlisting}
# Write a string, or sentence to a file. Using only one ">" will OVERWRITE the
# file if it exists, so be careful. Using two ">>" will append to the file
echo "test" >> a.txt 

# You can use the output from one command as the input for another.
# The following command will print ONLY THOSE LINES of a file, that contain
# a specified string.
cat "File.txt" | grep "string"
# If you only want the first line in which it occurs, just pipe it again
cat "File.txt" | grep "string" | head 1
\end{lstlisting}

\section{Program startup}
Linux has startup files, that load everytime a certain program is started.
\textbf{This is very useful}. For example \verb`.bashrc` is loaded, everytime
you open a terminal. If you want to tell your terminal, in which paths it has to
look for programs, this is where you would do it. Other examples are:
\begin{lstlisting}
/home/yourname/.profile		    # Loads on startup
/home/yourname/.vimrc		    # Loads on starting vim
/home/yourname/.config/i3/config    # Loads, when i3 is started
\end{lstlisting}


\section{Common Problems}
\subsection{wpa\_supplicant}
For automatic network connection make sure that the default interface 
(in this case \verb`wlp3s0` is configured in your \verb`/etc/network/interfaces`
):
\begin{lstlisting}
# The primary network interface
# This makes sure the interface is configured on startup
allow-hotplug wlp3s0
iface wlp3s0 inet
# Give a more verbose output for debugging dhcp
wpa-debug-level 3
# Path to your config file. Make sure it's correct
wpa-conf /etc/wpa_supplicant/wpa_supplicant.conf
\end{lstlisting}

\subsection{Fonts}
Installing fonts is difficult. But it doesn't have to be. Just make sure to
choose one directory (and only one) to include newly installed fonts from.
The default is \\ \verb`/usr/share/fonts/`. Put all fonts you want to install there
and run \verb'fc-cache' afterwards. Then, on startup, tell the system where the fonts
are by putting the following line into your \verb`/home/yourname/.config/i3/config`:
\begin{lstlisting}
exex xset +fp /usr/share/fonts/
\end{lstlisting}

\subsection{Pc Speaker (beep!)}
The Speaker can be quite annoying. To disable it, simply add two lines to \\
\verb`/etc/modprobe.d/blacklist.conf` (the \verb`.conf` is important!!)
\begin{lstlisting}
blacklist pcspkr
blacklist snd_pcsp
\end{lstlisting}

\subsection{Backlight}
Configuring xbacklight to change screen brightness is not straightforward.
You'll need to make some changes to \verb`/etc/X11/xorg.conf`.
If you get the "No outputs have backlight property" error, it is because
xrandr/xbacklight does not choose the right directory in \verb'/sys/class/backlight'.
You can specify the directory by setting the Backlight option of the device
section in xorg.conf. For instance, if the name of the directory is
\verb'intel_backlight', the device section can be configured as follows:
\begin{lstlisting}
/etc/X11/xorg.conf
-------------------
Section "Device"
    Identifier  "Card0"
    Driver      "intel"
    Option      "Backlight"  "intel_backlight"
EndSection
\end{lstlisting}
After a reboot, you should be able to use the normal commands
\verb`xbacklight -dec 20` etc.

\subsection{Copy to clipboard in vim}
It \textbf{IS} possible to yank lines to the system clipboard in vim.
Add the following lines to your \verb`.vimrc` 
\begin{lstlisting}
"yank to clipboard
if has("clipboard")
    set clipboard=unnamed " copy to the system clipboard

    if has("unnamedplus") " X11 support
	set clipboard+=unnamedplus
    endif
endif
" paste from clipboard
vnoremap <C-c> :w !pbcopy<CR><CR> noremap <C-v> :r !pbpaste<CR><CR>
" }}}
\end{lstlisting}
Then install \verb`vim-gtk` through your package-manager.

\section{Stuff i just cant seem to remember. And tips}
\begin{itemize}
\item 
Pavucontrol for microphone levels: \verb`./ts3client_runscript.sh` \\
To create bootable usb-drive:
\begin{lstlisting}
sudo dd bs=4M if=/path/to/iso of=/dev/sd$$ status=progress && sync # $$
\end{lstlisting}
\item
To create FAT32 Filesystem use
\begin{lstlisting}
mkfs -t vfat /dev/sd
\end{lstlisting}
\item
Copy stuff to clipboard
\begin{lstlisting}
echo "test" | tr -d '\n' | xclip -selection clipboard/primary
\end{lstlisting}
\item
Colors of urxvt
\begin{lstlisting}
color0	(black)	= Black
color1	(red)	= Red3
color2	(green)	= Green3
color3	(yellow)    = Yellow3
color4	(blue)	= Blue3
color5	(magenta)   = Magenta3
color6	(cyan)	= Cyan3
color7	(white)	= AntiqueWhite
color8	(bright black)	= Grey25
color9	(bright red)	= Red
color10	(bright green)	= Green
color11	(bright yellow)	= Yellow
color12	(bright blue)	= Blue
color13	(bright magenta)    = Magenta
color14	(bright cyan)	= Cyan
color15	(bright white)	= White
\end{lstlisting}
\item
For large screens i recommend adding this to \verb`.Xresources`
\begin{lstlisting}
Urxvt*font: xft:consolas:size=12
\end{lstlisting}
\item
The default font (in case you want to revert is \verb`Urxvt*font: 6x13` \\
To configure mouse speed:
\begin{lstlisting}
xinput --list			    # Lists all devices (mouse is 10)
xinput --list-props 10		    # Find out Transf. Matrix ID (140)
xinput set-prop 10 140 .5 0 0 0 .5 0 0 0 1  # Set mouse speed to 50%
\end{lstlisting}
\item
To connect Apple devices, you need to configure the USB protocol. \\
Identify the usbmuxd process: Open up a terminal and type,
\begin{lstlisting}
ps aux | grep usbmux
\end{lstlisting}
which should return an output like
\begin{lstlisting}
    usbmux    6781  0.0  0.0 230120  6584 ?        Sl
    09:30   0:00 /usr/sbin/usbmuxd -u -U usbmux
\end{lstlisting}
Kill usbmuxd: The highlighted number (6781) above is the process id/number for
usbmuxd (unique for all systems and the state of a system). What we want to do
is to kill this process and restart it. This can be done by the following
command in the terminal,
\begin{lstlisting}
sudo kill -9 6781
\end{lstlisting}
restart usbmuxd: With usbmuxd dead we need to manually start it again in
order for Ubuntu to one again recognize the iPhone/iPad/iPod. Bring up the
terminal once again and type,
\begin{lstlisting}
sudo usbmuxd -u -U usbmux
\end{lstlisting}
\item
Setup spicetify commands
\begin{lstlisting}
./spicetify config current_theme THEME_NAME # Choose Sweet for sweet colors
\end{lstlisting}
\end{itemize}
\end{document}
